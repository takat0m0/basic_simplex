% -*- coding:utf-8 -*-

\documentclass[11pt, a4]{article}
\usepackage{amsmath}
\usepackage{amsfonts}
\usepackage{ascmac}
\usepackage{algorithmic}
\usepackage{algorithm}

\title{simplex法の基礎}

\begin{document}
\maketitle

\section{基礎事項}
この section では,線形計画問題に関する基礎事項を述べる.入りとして,
数理計画問題の分類からスタートし,simplex 法の立ち位置について述べる.
そののちに,以降の準備として,線形計画問題の基本形や基礎用語について述べる.

\subsection{simplex 法の立ち位置}
一般に数理計画問題は以下のように表すことができる.

\begin{eqnarray}
  &min_{x, y}&\  f(x, y)\nonumber\\
  &s.t.&\ g_i(x, y) \leq 0\ (i \in \{1, \cdots, m\})\nonumber\\
  &\ & x\in \mathbb{R}^n, y \in \mathbb{Z}^{\ell}\ .
  \label{very_general_form}
\end{eqnarray}

日本語に直すと,
「コスト $f(x, y)$ を最小にする連続変数 $x$ と整数変数 $y$の組み合わせを見つけたい.
  ただし,$m$ 個の条件 $g_k(x,y) \leq 0$ を全て満足する $x, y$ でなければならない」
となる.
言葉だけの問題だが,コストのことを数理計画では目的関数(objective) と呼び,
条件のことを制約(costraint) と呼ぶので,今後はこの言葉を使っていくものとする.

さて,(\ref{very_general_form})式が数理計画問題の最も一般的な問題であるが,
まずはこれを分類する.分類の仕方は簡単で,以下の三つの観点で分類される.
\begin{itemize}
\item 整数変数があるかどうか.同じだが $\ell = 0$ or $\ell\neq 0$か.
\item 目的関数 $f(x, y)$ が linear か quadoratic か nonlinear か.
\item 制約式 $g_k(x, y)$ が linear か nonlinear か.
\end{itemize}

具体的に分類すると以下のようになり,それぞれ以下のような名前が付いている.
\begin{itemize}
\item 整数変数がない($\ell = 0$).
  \begin{itemize}
  \item 目的関数と制約式全てが linear $\rightarrow$ Linear Programming(LP)
  \item 目的関数が quadratic で制約式全てが linear $\rightarrow$ Quadoratic Programming(QP)
  \item それ以外 $\rightarrow$ Non-linear Programming(NLP)
  \end{itemize}
\item 整数変数がある($\ell \neq 0$).
  \begin{itemize}
  \item 目的関数と制約式全てが linear $\rightarrow$ Mixed Integer Linear Programming(MILP)
  \item 目的関数が quadratic で制約式全てが linear $\rightarrow$ Mixed Integer quadoratic Programming(MIQP)
  \item それ以外 $\rightarrow$ Mixed Intger Non-linear Programming(MINLP)
  \end{itemize}
\end{itemize}
このうちで,LP, QP, NLP, MILP, MIQPについては一般的なアルゴリズムが知られている.
つまり,数式に落とすことさえできれば,とりあえず解くこと自体は可能である
\footnote{勿論,十分高速に解けるかどうかは微妙である.}.
この中で,LP についてはいくつか効率的なアルゴリズムが知られているが,
そのうちの一つが simplex 法であり,今回紹介するアルゴリズムである.

\subsection{線形計画問題の基本形}
先に述べたように,simplex 法は LP を解くアルゴリズムであるが,
特に以下の形の問題を解くアルゴリズムである.

\begin{eqnarray}
  &\min_{x}&\ c^Tx\nonumber\\
  &s.t.&\ \sum_jA_{ij}x_j = b_i\ (i \in \{1, \cdots, m\}) , \nonumber\\
  &\ &\ x\geq 0, x \in \mathbb{R}^n
  \label{general_LP}
\end{eqnarray}
勿論,$c \in \mathbb{R}^n$, $A \in \mathbb{R}^{m\times n}$, $b\in \mathbb{R}^m$ は
constant な parameter で,解きたい問題に応じて与えるものである.
なお,この形で書く場合,通常 $m < n$ を仮定することが常である.
例えば $m = n$ かつ $A$ が full rank であれば,(\ref{general_LP}) 式はもうすでに解けているので
\footnote{勿論,これはこの形で書いた場合.不等式制約が残っている場合には $m < n$ を満たす必然性は存在しない}.

このような書き方をすると,「おいおい,これは一般的な形なのかよ?」と感じる人も多いかと思うが,
実はこれで一般的な形である.以下ではそのことを確認する.

まず,今回は LP を考えている,つまり,objective も constraint も線形なので,
大雑把には (\ref{general_LP})式のようになるのは想像できるかと思う.
ということで気になるのは以下の三点かと思われる.
\begin{enumerate}
\item objective 最大化はできないの?
\item $x \geq 0$ は限定しすぎでは?
\item 等号制約しか考えれないの?  
\end{enumerate}

まずは,1 から.これについては簡単で,$f(x)$ 最大化は $-f(x)$ の最小化と思えば良いので,
最小化問題に帰着させることができる.

続いて 2 について.これも比較的簡単で $x \in \mathbb{R}^n$ については,
二つのpostive な変数 $x^{\prime}, x^{\prime \prime}$ $(x^{\prime} \geq 0, x^{\prime\prime}\geq 0)$
を使って$x = x^{\prime} - x^{\prime\prime}$ と表してやれば
やっぱり (\ref{general_LP}) 式の形に帰着できる.

最後に 3 について.例えば,ある $i$ については,
$\sum_j A_{ij} x_j \geq b_i$ という不等式制約であったとする.
この制約式は補助変数 $s_i$ を導入すると以下のように書き換えることができる.
\begin{equation}
  \sum_j A_{ij} x_j - s_i = b_i, s_i \geq 0\ .
\end{equation}
よって,不等式制約も (\ref{general_LP}) 式の形にまとめて書くことができる.
なお,この等式が成り立つと思うと,
元の不等式制約を見てやればわかるように,
$s_i$ がその不等式に関する$x$ の「余裕度」を表している.


以上より,(\ref{general_LP}) 式は LP の一般的な形であり,
これを解くことができる simplex 法は LP の一般的な解法の一つと言える.

\subsection{基底解, 実行可能基底解}
(今回は出てくる行列がとりあえず full rank だと思って話をします...そうでない場合はまたいづれ...)

この subsection でも,これまで通り,変数の数を $n$,制約式の数を $m$ とする.
勿論,これまで述べたように $n > m$ とする.

(\ref{general_LP})の制約式 $Ax = b$ について考える.$n > m$ であるから,解くことはできない.
が,$n$ 個の変数のうち $n - m$個を選び,その変数を $0$ としてやれば $Ax = b$ を満たすような解を得ることができる.このような解を基底解と呼ぶ.
ただし,基底解は $Ax = b$ しか見ていないので,
$x \geq 0$ を満たすかどうかは不明である.基底解のうちで $x \geq 0$ も満たすようなものを
実行可能基底解と呼ぶ.

以上が基底解と実行可能基底解の言葉での定義となってしまうが,
simplex 法の説明にも使うので,数式でも説明しておく.

まず,変数の index 集合 $\{1, \cdots, n\}$を $m$ 個と $n - m$ 個の集合に分割する.
前者を $B_{index}$, 後者を $N_{index}$ と呼ぶことにする
\footnote{Basic と Non-Basic の略.simplex 法を見ると,個人的には Non-Basic の方が Basic な感じがするが...}.
すると,
\begin{eqnarray}
  b_i = \sum_j A_{ij} x_j = \sum_{j\in B_{index}}A_{ij}x_j + \sum_{j \in N_{index}}A_{ij}x_j
\end{eqnarray}
と書くことができる.この右辺を以下のように書くことにする.
\begin{equation}
  b = Bx_B + Nx_N\ .
\end{equation}
これは,$A$ の列を適当に並び替えた上で $A = [B|N] (B \in \mathbb{R}^{m\times m}, N \in \mathbb{R}^{m\times n - m})$と分割し,
さらに $x$ を適当に並び替えて $x = (x_B, x_N)^{T}(x_B \in \mathbb{R}^m, x_N \in \mathbb{R}^{n -m})$と分割して
\begin{eqnarray}
  b = Ax = \left[
    \begin{array}{c|c}
      B&N
    \end{array}
    \right]
  \left[
    \begin{array}{c}
      x_B\\
      x_N
    \end{array}
    \right] = Bx_B + N x_N
  \label{BN_decompose}
\end{eqnarray}
と考えれば同じものであることが確認できる.

このような分割(\ref{BN_decompose})を考えれば,
基底解は $x_N = 0$ で特徴付けられるので,基底解は
\begin{equation}
  x_B = B^{-1}b, x_N = 0
\end{equation}
と書くことができる.さらに,$x_B \geq 0$ つまり $B^{-1} b\geq 0$ が成立する場合(あるいは成立するような分割の場合)にそれを実行可能基底解と呼ぶ.

\subsubsection{基底解,実行可能基底解の例}
ここまで,LP の一般的な形と基底解,実行可能基底解と言葉ばかり並べてきたので,
一つ例を紹介する.

以下のような制約を考えてみよう.
\begin{eqnarray}
  3x_1 + 2x_2 &\leq& 12\nonumber\\
  x_1 + 2x_2 &\leq& 8\nonumber\\
  x_i&\geq&0\ .
  \label{example}
\end{eqnarray}
こいつをイコール制約に直すと
\begin{eqnarray}
  3x_1 + 2x_2 + s_1 &=& 12\nonumber\\
  x_1 + 2x_2 + s_2&=& 8\nonumber\\
  x_i, s_i&\geq&0
\end{eqnarray}
となる.これを行列表記に直すと
\begin{eqnarray}
  \left[
  \begin{array}{cccc}
    3&2&1&0\\
    1&2&0&1
  \end{array}
  \right]
  \left[
  \begin{array}{c}
    x_1\\
    x_2\\
    s_1\\
    s_2
  \end{array}
  \right]
  =
  \left[
  \begin{array}{c}
    12\\
    8
  \end{array}
  \right]
\end{eqnarray}
となるので,
\begin{eqnarray}
  A =
  \left[
  \begin{array}{cccc}
    3&2&1&0\\
    1&2&0&1
  \end{array}
  \right]
  ,
  b =
  \left[
  \begin{array}{c}
    12\\
    8
  \end{array}
  \right]
\end{eqnarray}
であることがわかる.

この系は変数の数 $n = 4$ であり,
制約式の数 $m = 2$であるので,
基底解を作るための$0$に選べる変数の数は $n - m = 2$ 個だから
基底解は ${}_{4}C_2 = 6$ 通りだけある.

\begin{enumerate}
\item $N_{index} = \{x_1, x_2\}, B_{index} = \{s_1, s_2\}$
  
\begin{eqnarray}
  B =
  \left[
  \begin{array}{cc}
    1&0\\
    0&1
  \end{array}
  \right]
  ,
  N =
  \left[
  \begin{array}{cc}
    3&2\\
    1&2
  \end{array}
  \right]
\end{eqnarray}
このとき,$x_1 = x_2 = 0$ であるから,$s = b$ となる.元の変数の空間$(x_1, x_2)$ で見れば原点である.図からも$s$ が「余裕度」であることがわかると思う.
  
\item $N_{index} = \{x_1, s_1\}, B_{index} = \{x_2, s_2\}$
\begin{eqnarray}
  B =
  \left[
  \begin{array}{cc}
    3&1\\
    1&0
  \end{array}
  \right]
  ,
  N =
  \left[
  \begin{array}{cc}
    2&0\\
    2&1
  \end{array}
  \right]
\end{eqnarray}
このとき,基底解は$(x_1, s_1) = x_B = B^{-1}b = (8, -12)^T$となっている.
勿論 $x_2 = s_2 = 0$ である.
これを元の $(x_1, x_2)$ 空間で考えてみる.まず $x_2 = 0$ である.
さらに,$s_2 = 0$ なので(\ref{example})の二つ目の不等式は等式となっている.
そのため,この基底解は $x_2 = 0$と $x_1 + 2 x_2 = 8$との交点に対応している.
(以降,同じような話の場合は答えだけ書く.)

\item $N_{index} = \{x_1, s_2\}, B_{index} = \{x_2, s_1\}$
\begin{eqnarray}
  B =
  \left[
  \begin{array}{cc}
    3&0\\
    1&1
  \end{array}
  \right]
  ,
  N =
  \left[
  \begin{array}{cc}
    2&1\\
    2&0
  \end{array}
  \right]
\end{eqnarray}
このとき,基底解は$(x_1, s_2) = x_B = B^{-1}b = (4, 4)^T$となっている.
勿論 $x_2 = s_1 = 0$ である.

\item $N_{index} = \{x_2, s_1\}, B_{index} = \{x_1, s_2\}$
  \begin{eqnarray}
  B =
  \left[
  \begin{array}{cc}
    2&1\\
    2&0
  \end{array}
  \right]
  ,    
  N =
  \left[
  \begin{array}{cc}
    3&0\\
    1&1
  \end{array}
  \right]
\end{eqnarray}
このとき,基底解は$(x_2, s_1) = x_B = B^{-1}b = (4, 4)^T$となっている.
勿論 $x_1 = s_2 = 0$ である.

\item $N_{index} = \{x_2, s_2\}, B_{index} = \{x_1, s_1\}$
  \begin{eqnarray}
  B =
  \left[
  \begin{array}{cc}
    2&0\\
    2&1
  \end{array}
  \right]
  ,    
  N =
  \left[
  \begin{array}{cc}
    3&1\\
    1&0
  \end{array}
  \right]
\end{eqnarray}
このとき,基底解は$(x_2, s_2) = x_B = B^{-1}b = (6, -4)^T$となっている.
勿論 $x_1 = s_1 = 0$ である.

\item $N_{index} = \{s_1, s_2\}, B_{index} = \{x_1, x_2\}$
  \begin{eqnarray}
  B =
  \left[
  \begin{array}{cc}
    3&2\\
    1&2
  \end{array}
  \right]
  ,    
  N =
  \left[
  \begin{array}{cc}
    1&0\\
    0&1
  \end{array}
  \right]
\end{eqnarray}
このとき,基底解は$(x_1, x_2) = x_B = B^{-1}b = (2, 3)^T$となっている.
勿論 $s_1 = s_2 = 0$ である.
これは $(x_1, x_2)$平面上で (\ref{example}) の二つの不等式が共に等号が成り立つ線の交点に対応しているが,
それは $s_1 = s_2 = 0$と符号している.
\end{enumerate}

この例からもわかるように,実行可能基底解は,
元の空間で見ると,不等式制約が定める凸集合の端点に相当している.

\section{simplex 法}

準備が整ったので,LP の一般的な解法である simplex 法について述べる.
ここで見るように,simplex 法は
「実行可能基底解を input として,最適な実行可能基底解を output する」algorithm である.
これを聞くと「input となる実行可能規定解はどう得るんだ?」と思う人も多いかと思うが,
それについての答えは次の二段階 simplex のsection で述べる.
(この状況ではちょっと tortological に聞こえると思うが,
input の実行可能基底解を作るのにも simplex を使うため.)

\subsection{simplex 法の気分}
今回考えている問題は線形な問題である.
ということは,必ず「領域の端」で最適解を取るはずである.
つまり,前 section の言葉を使えば
「実行可能基底解のいずれかが,最適解を与える」となる.

さらに,良いことに,今回の objective は線形なのである.
つまり,local minimum が存在しない.そのため,
「今見ている実行可能基底解の『隣』の実行可能基底解で,
objective が下がるものを探し続ければ最適解に辿りつく」
という戦略が思いつく.
そして,実は simplex 法が行っていることはほぼこれである.
以下では,simplex 法の一般的な algorithm を述べた上で,
この気分が正しいことを例(といっても添付の jupyter notebook だが)で確認する.

\subsection{simplex 法}
\subsubsection{最適性の条件, pricing rule}
ある実行可能基底解 $x_B = B^{-1}b \geq 0, x_N = 0$ が与えられたとする.
このときに,この解の近くで,制約を満たすように $x_N$ を non-zero にしていくことを考える.
そうした場合に,やっぱり $x_N = 0$ が objective を最小に与える,
つまり,与えられた実行可能基底解が最適である条件を考える.


一般的な線形計画問題は
\begin{eqnarray}
  &\min_{x}&\ c^Tx\nonumber\\
  &s.t.&\ Ax = b,\ x\geq 0, x \in \mathbb{R}^n
\end{eqnarray}
こうであるが,与えられた実行可能基底解を基礎として,制約式を分解すると
\begin{equation}
  Bx_B + Nx_N = b \Leftrightarrow x_B = B^{-1}(b - Nx_N)
\end{equation}
となるので,これを元の問題に代入すると,objective が
\begin{equation}
  c^T x = c_B^Tx_B + c_N^T x_N = c_B^TB^{-1}(b - Nx_N) + c_N^Tx_N
\end{equation}
ということに注意すると
\begin{eqnarray}
  &\min_{x_N}&\ c_B^TB^{-1}b + (c_N - N^T B^{-1 T}c_B)^Tx_N\nonumber\\
  &s.t.& B^{-1}(b - Nx_N)\geq 0, x_N\geq 0
  \label{simplex_LP}
\end{eqnarray}
という $x_N$ だけの問題に落とすことができる.ちなみに,objective の第一項は
$x_N = 0$ の場合,つまり与えられた実行可能基底解の objective の値に他ならない.

今後使うので,vector 
\begin{equation}
  \rho = c_N - N^TB^{-1 T}c_B
  \label{def_rho}
\end{equation}
を定義しておく.勿論 (\ref{simplex_LP}) の第二項である.
ここで,特に $x_N \geq 0$ と この$\rho$に注目してみる.
もし $\rho > 0$だとしよう.
すると,(\ref{simplex_LP})は明らかに $x_N = 0$ のときに最小値を取ることがわかる.
つまり,この場合に,与えられた実行可能基底解が最適で,最適値 $c_B^TB^{-1}b$を取る.

それに対して $\rho_j < 0 (j \in N_{index})$ となるような $j$ があった場合は
どうかというと,
そのような $j$ に対して,$x_j$ を $0$ から大きくすることで objective を下げることができる.
つまり,このようなケースでは,与えられた実行可能基底解が最適ではなく,もっと objective を下げることができる.


\subsubsection{pricing, ratio test}
前 subsubsection で$\rho_j < 0 (j \in N_{index})$ となるような $j$ があった場合は,
例えばそのような $j$ から一つ選び,$x_j$ を大きくすると objective を下げることができることを確認した.
以下ではそのような変数を一つ選んだ場合にどうなるかをみる.
なお,このような変数の選択を pricing とか pricing rule とか呼ばれている
\footnote{pricing rule には,一応色々な方法が考案されている.}.

このように選択された $j$ を $k \in N_{index}$ と書くことにしよう.
$x_N$ として,$k$ にしか成分がないような状況を考える.
$x_k = \xi$ と書けば, (\ref{simplex_LP})は
\begin{eqnarray}
  &\min_{\xi}&\ c_B^TB^{-1}b + \rho_k\xi\nonumber\\
  &s.t.&\ \sum_{\ell}(B^{-1})_{j\ell}b_{\ell} \geq \sum_{\ell} (B^{-1})_{j\ell} A_{\ell k}\xi\ (j \in B_{index}),\nonumber\\
  &\ & \xi\geq 0
    \label{simplex_LP_prime}
\end{eqnarray}
と書き換えることができる.勿論,これは $x_k$ しか動かしていないローカルな最適化問題で,
もとの最適化問題とは異なる.

このローカルな最適化問題を考えるが,実はこれは簡単に解くことができる.
objetive を見れば$\xi$ は大きければ大きいほど良いが,
制約式を見ると頭打ちに合っていることがわかる.
具体的には
\begin{equation}
  {\bar b}_j = \sum_{\ell}(B^{-1})_{j\ell}b_{\ell} > 0,\  y_j =  \sum_{\ell} (B^{-1})_{j\ell} A_{\ell k}
  \label{b_bar}
\end{equation}
と置いた場合に以下の $\theta$ までは増やすことができる\footnote{なお,もし全て$y_i < 0$ならば,このローカルな問題,ひいては元々の問題の答えは非有界である.}.
\begin{eqnarray}
  j_{min} = \arg \min({\bar b}_j/y_j | y_j > 0),\  \theta = {\bar b}_{j_{min}}/y_{j_{min}}
  \label{i_min}
\end{eqnarray}
勿論 ${\bar b} > 0$ は実行可能基底解を持ってきていることによる.
よって,ローカルな最適化問題は $\xi = \theta$ のときに最適解を取る.

さて,与えられた実行可能基底解の周りについて考えていたが,
この実行可能基底解で$x_N$ としてこの $k$ の成分が $\theta$ になったときどうなるか,というと
$x_B^{mod} = B^{-1}(b - Nx_N)$ に代入して計算すると,
\begin{equation}
  x_B^{mod}{}_j = {\bar b}_j - y_j\theta
\end{equation}
となるが,$\theta$ の選び方から,
$x_B^{mod}{}_j > 0 (j \neq j_{min})$かつ $x_B^{mod}{}_{j_{min}} = 0$である.
勿論 $x_k = \theta \neq 0$ である.
よって,ローカルな最適化問題の最適解は,元の実行可能基底解の $B_{index}$ と $N_{index}$ を
利用して書けば,$B^{new}_{index} = (B_{index}\backslash \{j_{min}\}) \cup \{k\}$, 
$N^{new}_{index} = (N_{index}\backslash \{k\}) \cup \{j_{min}\}$ で書かれる実行可能基底解である.
勿論この新しい実行可能基底解の objective は元の実行可能規定解の objective より
小さな値を取っている.
さらに,新しいものも実行可能基底解である.よって同じ操作,
つまりローカルな問題(\ref{simplex_LP_prime})を定義して解くこと,
を繰り返すことでどんどん objective を下げていくことができる.
そして,LP なので,このように下げていっても local minimum にはまることはないので,
これで最適解に辿りつくことができるというわけである.

これが simplex 法の根本的な考え方であるが,
念のため一つだけ指摘をしておくと,
実行可能基底解が与えられ pricing で $k$ を選んだのちのローカルな
最適化問題(\ref{simplex_LP_prime})であるが,この問題自体は
 (\ref{b_bar})を計算して(\ref{i_min})を確認するだけの簡単な問題である.
特に(\ref{i_min})から ratio test と呼ばれてる作業だが,
ほぼ簡単な代数操作をするだけである.
よって simplex 法は突き詰めれば,
「pricing と ratio test を繰り返す」だけの algorithm とも言える.

\subsubsection{algorithm}
以上まとめると simplex 法の algorithm は以下の通りである.
\begin{algorithm}
\begin{algorithmic}
  \REQUIRE 実行可能基底解
  \ENSURE 最適な実行可能基底解

  現在の実行可能規定解 $\leftarrow$ inpout 実行可能基底解
  \LOOP
  \STATE 現在の実行可能基底解と(\ref{def_rho}) に基づいて $\rho$ を計算する.
  \IF{$\rho > 0$}
  \STATE return 現在の実行可能基底解解
  \ENDIF
  \STATE -- pricing --
  \STATE なんらかのルールで pricing を行って $k$ を選ぶ.
  \STATE -- ratio test --
  \STATE (\ref{b_bar})を計算
  \STATE (\ref{i_min})によって $B_{index}\rightarrow N_{index}$となる変数 $j_{min}$を選択
  \STATE -- 実行可能基底解の更新 --
  \STATE $k$, $j_{min}$ を基に現在の実行可能基底解を更新
  \ENDLOOP
\end{algorithmic}
\end{algorithm}
\subsection{例}
以下の問題を考える.

これを解いていく過程が添付の jupyter notebook となっている.
これを見ていただければ,

\section{二段階 simplex 法}

以上に見たように simplex 法は
「実行可能基底解を input として,最適実行可能基底解を output する」algorithm であった.
ここで勿論気になるのは「input である実行可能規定解をどう作るのか」である.
実は,この input も simplex 法で作ることができる.

まずは,毎度お馴染み LP の一般的な問題からスタートする.
\begin{eqnarray}
  &\min_{x}&\ c^Tx\nonumber\\
  &s.t.&\ \sum_jA_{ij}x_j = b_i\ (i \in \{1, \cdots, m\}) , \nonumber\\
  &\ &\ x\geq 0, x \in \mathbb{R}^n
  \label{org_LP}
\end{eqnarray}

この問題に対して以下のような問題を考えてみよう.
\begin{eqnarray}
  &\min_{x, s}&\ \sum_{i = 1}^m s_i \nonumber\\
  &s.t.&\ \sum_jA_{ij}x_j + sign(b_i) s_i = b_i\ (i \in \{1, \cdots, m\}) , \nonumber\\
  &\ &\ x\geq 0, s\geq 0, x \in \mathbb{R}^n, s \in \mathbb{R}^m\ ,
  \label{modified_LP}
\end{eqnarray}
ここに $sign(a) = a/|a|$ である.
この問題の意味は次の通りである.

\begin{itembox}[l]{(\ref{modified_LP})の気分}
$x$ が何か与えられたときに,各制約について $s$ はその破れ具合を表している.
今回の objective は破れ具合の和 $\sum_i s_i$ であるから,
この問題を解き,その結果が objective = 0 な解だった場合には,
元問題の実行可能解が得られる.
\end{itembox}

(\ref{modified_LP}) 式 を simplex 法で解いてみよう.
simplex 法は「実行可能基底解を input として,最適実行可能基底解を output する」algorithm であった.
(\ref{modified_LP}) 式はありがたいことに,以下の自明な実行可能基底解が存在する.
\begin{equation}
  x_j = 0, s_i = |b_i|\ .
\end{equation}
つまり,(\ref{modified_LP})式の simplex 法の input は問題なく用意できる.
なので,こいつを input にして simplex 法を回すことができる.
その output は何かと言うと (\ref{modified_LP})式の最適な実行可能基底解である.
さてその解であるが,
(\ref{modified_LP})の目的関数が元問題の制約式の破れ具合の $\sum_i s_i$ であることから,
もし (\ref{org_LP})が infeasible でなかった場合は,(\ref{modified_LP})の最適解 $(x, s)$は
$s_i = 0$ であるような実行可能基底解であるはずである.
そして $s_i = 0$ であることから,(\ref{modified_LP})を見ればわかるように,
そのような実行可能基底解は,元問題 (\ref{org_LP}) の実行可能基底解となっている.
よって(\ref{modified_LP})を simplex 法で解くことによって (\ref{org_LP})の
input を作成することができる.

以上のように,input となる実行可能基底解 も simplex 法で作ることができるので,
結果的に二回 simplex 法を解くことで一般的な LP を解くことが可能である.
そのためこのような解き方を 二段階 simplex 法と呼ばれている.
まとめれば,二段階 simplex 法を利用することで一般的な LP を input として,
最適解を output できる.

\section{まとめ}
この資料では,一般的な LP や LP にまつわる用語から始め,
simplex 法や二段階 simplex 法について解説をした.
しかし,
\begin{itemize}
\item 実は simplex はちょっと遅い(内点法の法が一般的には速い).
\item また,最適解を切り落とすような制約式の追加に弱い(dual simplex であれば問題ない.
この性質があるために MILP の一般的解法である branch and bound では dual simplex が使われている.).
\end{itemize}
といった問題がある.これらの解決は今後このゼミでなされ続けるはずである.
\end{document}
